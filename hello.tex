\documentclass[twocolumn]{article}

\setlength{\oddsidemargin}{-0.5in}
\setlength{\evensidemargin}{-0.5in}
\setlength{\topmargin}{0.0in}
\setlength{\headheight}{0in}
\setlength{\headsep}{0in}
\setlength{\textwidth}{7.0in}
\setlength{\textheight}{9.5in}
\setlength{\parindent}{0in}
\setlength{\parskip}{0.05in}
\setlength{\columnseprule}{0.3pt}
\usepackage{fancyvrb}
\usepackage{relsize}
\usepackage{hyperref}

\begin{document}

Name: \_\_\_\_\_\_\_\_\_\_\_\_\_\_\_\_\_\_\_\_\_\_\_\_\_\_\_\_

Directions: {\bf \Large Work only on this sheet} (on both sides, if
needed); do not turn in any supplementary sheets of paper. There is
actually plenty of room for your answers, as long as you organize
yourself BEFORE starting writing.

{\bf \Large Unless otherwise stated, give numerical answers as
expressions, e.g. $\frac{2}{3} \times 6 - 1.8$.  Do NOT use
calculators.}

{\bf 1.} Suppose the random vector $X = (X_1,X_2,X_3)'$ has mean
$(2.0,3.0,8.2)'$ and covariance matrix

\def\refto#1{\in@AmS,{#1}\if T\cresult@AmS\refto@AmS#1\end@AmS\else

  
\def\Let@AmS{\relax@AmS\iffalse{\fi\let\\=\cr\iffalse}\fi}

\begin{equation}
   \left (
   \begin{array}{ccc}
   1 & 0.4 & -0.2\\
   \  & 1 &  0.25 \\
   \  & \  & 3
   \end{array}
   \right )     
\end{equation}

\begin{itemize}

\item [(a)] (10) Fill in the three missing entries.

\item [(b)] (10) Find $Cov(X_1,X_3)$.

\item [(c)] (10) Find $\rho(X_2,X_3)$.

\item [(d)] (10) Find $Var(X_3)$.

\item [(e)] (15) Find the covariance matrix of $(X_1+X_2,X_2+X_3)'$.

\item [(f)] (15) If in addition we know that $X_1$ has a normal
distribution, find $P(1 < X_1 < 2.5)$, in terms of $\Phi()$.

\item [(g)] (15) Consider the random variable $W = X_1 + X_2$.  Which of
the following is true?
(i) $Var(W) = Var(X_1+X_2)$.
(ii) $Var(W) > Var(X_1+X_2)$.
(iii) $Var(W) < Var(X_1+X_2)$.
(iv) In order to determine which of the two variances is the larger one,
we would need to know whether the variables $X_i$ have a multivariate
normal distribution.
(v) $Var(X_1+X_2)$ doesn't exist.

\end{itemize}

{\bf 2.} (15) What is the (approximate) output of this R code:

\begin{Verbatim}[fontsize=\relsize{-2}]
count <- 0
for (i in 1:10000) {
   count1 <- 0
   count2 <- 0
   count3 <- 0
   for (j in 1:20) {
      x <- runif(1)
      if (x < 0.2) {
         count1 <- count1 + 1 
      } else if (x < 0.6) count2 <- count2 + 1 else 
              count3 <- count3 + 1
   }
   if (count1 == 9 && count2 == 2 && count3 == 9) count <- count + 1
}
cat(count/10000)
\end{Verbatim}

{\bf Solutions:}

{\bf 1a.}

\begin{equation}
   \left (
   \begin{array}{ccc}
   1 & 0.4 & -0.2\\
   0.4  & 1 &  0.25 \\
   -0.2  & 0.25  & 3
   \end{array}
   \right )     
\end{equation}

{\bf 1b.} -0.2

{\bf 1c.} $ \frac {0.25} {\sqrt{1} \sqrt{3}} $

{\bf 1d.} 3

{\bf 1e.}

$$
   \left (
   \begin{array}{ccc}
   1 & 1 & 0 \\
   0 & 1 & 1 
   \end{array}
   \right )     
   \left (
   \begin{array}{ccc}
   1 & 0.4 & -0.2\\
   0.4  & 1 &  0.25 \\
   -0.2  & 0.25  & 3
   \end{array}
   \right )     
   \left (
   \begin{array}{cc}
   1 & 0 \\
   1 & 1 \\
   0 & 1 
   \end{array}
   \right )     
$$

{\bf 1f.} $
\Phi(\frac{2.5-2.0}{1}) - 
\Phi(\frac{1-2.0}{1}) 
$

{\bf 1g.} (ii), by (3.29)

{\bf 2.}

$
\frac{20!}{9!^2 2!}
0.2^9 0.4^{11} 
$
$f(x)=x^2[$] normal text $g(x)=2$
\end{document}

